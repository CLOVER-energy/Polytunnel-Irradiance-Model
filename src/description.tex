\documentclass{report}
\usepackage[utf8]{inputenc}

\begin{document}

\section{Parameters description and geometry}



Figure 1. Geometry for polytunnel. The length of the polytunnel is aligned with y-axis coordinate in code, radius1 is aligned with x-axis in code and radius2 is aligned with z-axis in code.


Figure 2. Configuration of an OPV on the roof of polytunnel, in the example, the active layer has the same thickness, 100 nm (PTQ-10: 50 nm and Y6 : 50 nm). a) it is configure to have only a bilayer, while b) is configured to divide active layer in two consecutive bilayers. Multistack parameter must be an integer number.


 













Figure 3. Projection of Polytunnel Roof on xy plane. The separation of every PV module is controlled by cell_gap variable, every PV module has dimensions of cell_thickness in from the side oriented with y-axis. Variable initial_cell_gap controls the distance between the edge of polytunnel and first PV module. 
\section{Simulation Excecution}

The script can be executed in two different ways: by parsing the name of the variables or by using a CSV file with two columns, “Parameter” and “Value”. 

\section{Running using argparse}

python main.py --start_time_str 2024-06-06T00:00:00Z --end_time_str 2024-06-06T23:59:59Z --latitude 51.249814 --longitude 0.347779 --res_minutes 15 --length 2.5 --radius1 2.5 --radius2 1.5 --xy_angle 0 --z_angle 0 --transmissivity 1 --material_list ag ag ag ag ag ag --material_thick 80 80 80 80 80 80 --multistack 1 --cell_thickness 0.35 --cell_gap 2.8 --initial_cell_gap 2.1 --res_meshgrid 0.35 

Running using a external csv file with the name of the parameters and their values:

python main.py –csv /path/to/parameters_simulation.csv

if the file parameters_simulation.csv is in the same directory as the main.py, then the execution should be:

python main.py –csv parameters_simulation.csv

The content of parameters_simulation.csv can be organized as follows, a column called “Parameter” contains the name of variables used by the script main.py to execute the simulation, and a column called  “Value”, which  contains the respective value for every parameter:

Parameter,Value
start_time_str,2024-06-06T00:00:00Z           
end_time_str,2024-06-06T23:59:59Z
latitude,51.249814
longitude,0.347779
res_minutes,15
length,2.5
radius1,2.5
radius2,1.5
xy_angle,0
z_angle,0
transmissivity,1
material_list,ag ag ag ag ag ag
material_thick,80 80 80 80 80 80
multistack,1
cell_thickness,0.35
cell_gap,2.8
initial_cell_gap,2.1
res_meshgrid,0.35

NOTE: Considerations in PV modules configuration:
In the previous example, the simulation considers a a silver module with thickness of 480 nm, that will block all the light in the positions of the PV modules. It is necessary to divide the material_list and material_thick in at least 6 segments, since it is configured to consider the structure of an OPV solar cell module, like the one is presented in Figure 2. 

\subsection{Running a simulation without PV modules}:

To simulate the case without PV modules in the roof of Polytunnel, you need to use the next set of variables:

cell_thickness,0.0
cell_gap, 0.0
initial_cell_gap,0.0

\section{Spectral Sensitivity of BF5 Sensor}

The spectral response of BF5 Sunshine Sensor - Pyranometer  has been included in main.py script, in specific in lines:


Document with spectral data can be found in directory resources: “BF5-User-Manual-v2.2.pdf”. The spectral response of BF5 sunshine sensor can be found in page 34. The data was extracted with 
Inkscape vector graphics editor.


\section{Possible future work}
Consider to use diffuse and direct components from pvlib module spectrum.spectrl2, currently the code considers direct intensity as “poa_global” in sun.py module:















However,  outputs from wrapper  spectrum.spectrl2 include:
    • wavelength
    • dni_extra
    • dhi
    • dni
    • poa_sky_diffuse
    • poa_ground_diffuse
    • poa_direct
    • poa_global



\section{Web Scrapping Data From Renewables Ninja INCLUDED}

A script to scrap data from renewables ninja is in directory ninjaData, and the name of the script is ninja_scrap_api.py, however, it is necessary to be edited to obtain the data from the place and dates you need. Additionally it requires a token, and  renewables ninja only allow you to extract 50 querys per user per hour. 



\end{document}